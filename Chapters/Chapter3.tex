% Chapter 3

\chapter{Literature review} % Main chapter title

\label{Chapter3} % For referencing the chapter elsewhere, use \ref{Chapter3} 

%----------------------------------------------------------------------------------------

\section{Creativity}
When looking at different definitions of creativity, it is quite abstract. Here are some definitions of creativity:

Wikipedia:
Creativity is a phenomenon whereby something new and somehow valuable is formed. The created item may be intangible (such as an idea, a scientific theory, a musical composition, or a joke) or a physical object (such as an invention, a literary work, or a painting).

Google/Oxford:
the use of imagination or original ideas to create something; inventiveness.
"firms are keen to encourage creativity"
synonyms:
inventiveness, imagination, imaginativeness, innovation, innovativeness, originality, individuality; 
artistry, expressiveness, inspiration, vision, creative power, creative talent, creative gift, creative skill, resourcefulness, ingenuity, enterprise
"challenging objectives motivate staff and encourage creativity"

Dictionary.com
%noun
1.
the state or quality of being creative.
2.
the ability to transcend traditional ideas, rules, patterns, relationships, or the like, and to create meaningful new ideas, forms, methods, interpretations, etc.; originality, progressiveness, or imagination:
the need for creativity in modern industry; creativity in the performing arts.
3.
the process by which one utilizes creative ability:
Extensive reading stimulated his creativity.

Merriam-Webster:
Definition of creativity
1:  the quality of being creative
2:  the ability to create her artistic creativity

\subsection{How is creativity done?}
It seems hard to define specifically how creativity is actually done, since it can be done in many ways. The established literature seems to branch into different methods. Our approach will follow the theory of conceptual blending.

%----------------------------------------------------------------------------------------

\section{Computational creativity}


%----------------------------------------------------------------------------------------

\section{Concept blending}
Concept blending is a theory proposed by Gilles Fauconnier and Mark Turner. It works by projecting structure from input mental spaces to a separate, "blended" mental space. \parencite{reference3}
%According to \parencite{reference3}: “concept blending constitutes a cognitive process which allows for the combination of certain elements (and their relations) from originally distinct conceptual spaces into a new unified space combining these previously separate elements and allowing the performance of reasoning and inference over the combination.”

%----------------------------------------------------------------------------------------

\section{Analogy}

%Relasjonar er viktig i analogiar
%Analogiar er viktig for concept blending
%Generalisering?

%Dedre Gentner 1983

%Structure mapping engine is a direct mapping approach

%Generalization based models of analogy
%Smaling 2003
%%HDTP (Schmidt et al. 2014)

%Min løysing er delvis inspirert av analogi ved generalisering?

%----------------------------------------------------------------------------------------

\section{Amalgam}
One approach to concept blending is the use of amalgams. \parencite{Reference4}
An amalgam uses generalization of two concepts to make them compatible for blending through unification of the two generalizations. When parts of the concepts share a similar generalization, but are different, they are not unifiable. But they can unify with the \emph{least general generalization}. To make the concepts unifiable, we can therefore generalize corresponding pairs of parts between the concepts. For instance, when creating an amalgam of a \emph{red French vehicle} and a \emph{German minivan}, we can make them unifiable by generalizing the first to a \emph{red european vehicle}. When unifying we get a \emph{red german minivan}. Or if we generalize the second to a \emph{european minivan}, we can unify to a \emph{red french minivan}.
%An amalgam of two descriptions is a new description that contains parts from these two descriptions. (sitat frå Besold eller Ontanón and Plaza 2010?)

%Subsumption relasjon (hyponym subsuming hypernym)
%All informasjon i hypernymet er sant for hyponymet

%----------------------------------------------------------------------------------------

\section{The COINVENT Project}

%----------------------------