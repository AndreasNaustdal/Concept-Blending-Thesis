% Chapter 3

\chapter{Research methods and methodologies} % Main chapter title

\label{Chapter3} % For referencing the chapter elsewhere, use \ref{Chapter3} 

%----------------------------------------------------------------------------------------

\section{Personal Kanban}
In order to keep my work organized and productive, I used the agile method called Personal Kanban. To visualize the workload, Personal Kanban uses a value stream which are columns that divide the work based on the progress of each task. I used the simplest one that divides the work into three columns. The first is the project backlog which includes all the tasks that awaits production. The middle column is the tasks currently being worked on, which should be a limited number of tasks so that they are not getting stuck half-finished and distracting the task one is actually doing. The last column is the tasks that have been completely done so that there are no reason to be distracted by it anymore. \parencite{Reference1}

To keep a workflow where I can spend my energy on making sure the model, view and controller components is interacting well together, I limited my task number to 3. This way I could select one task from each component that depended on each other, while unfinished tasks would not be slowing down the workflow.\\
To review my progress, I had a daily evaluation inspired by the stand-up meeting used in XP \parencite{Reference2} where I answered the following questions:
\begin{enumerate}
\item What was accomplished yesterday?
\item What will be attempted today?
\item What problems are causing delays?
\end{enumerate}


%----------------------------------------------------------------------------------------

\section{Design Science}
This thesis follows the IT research framework by \citet{march1995design}. We have built and evaluated a method based on a concept blending algorithm.

\begin{figure}
\centering
\includegraphics[width=1\linewidth]{"Figures/design science figur"}
\caption{Research framework \parencite{march1995designfigure}}
\label{fig:design-science-figur}
\end{figure}


%----------------------------------------------------------------------------------------