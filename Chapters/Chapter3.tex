% Chapter 3

\chapter{Research methods and methodologies} % Main chapter title

\label{Chapter3} % For referencing the chapter elsewhere, use \ref{Chapter3} 

%----------------------------------------------------------------------------------------
In this chapter we describe our methods for development and research.

\section{Personal Kanban}
In order to keep my work organized and productive, I used the agile method called Personal Kanban. To visualize the workload, Personal Kanban uses a value stream which are columns that divide the work based on the progress of each task. I used the simplest one that divides the work into three columns. The first is the project backlog which includes all the tasks that awaits production. The middle column is the tasks currently being worked on, which should be a limited number of tasks so that they are not getting stuck half-finished and distracting the task one is actually doing. The last column is the tasks that have been completely done so that there are no reason to be distracted by it anymore. \parencite{benson2009}

To keep a workflow where I can spend my energy on making sure the model, view and controller components is interacting well together, I limited my task number to 3. This way I could select one task from each component that depended on each other, while unfinished tasks would not be slowing down the workflow.\\
To review my progress, I had a daily evaluation inspired by the stand-up meeting used in XP \parencite{wells1999} where I answered the following questions:
\begin{enumerate}
\item What was accomplished yesterday?
\item What will be attempted today?
\item What problems are causing delays?
\end{enumerate}


%----------------------------------------------------------------------------------------

\section{Research methods}

\citet{simon1996sciences} separate design science from natural science.
\citet{march1995design} state that research activities in design science and natural science has different intentions. Design science intends to build and evaluate artifacts, while natural science intends to theorize and justify theories.

\subsection{Design Science}


%Artifacts created in design science tries to achieve a goal. (TODO SITAT! H Simon) Our artifact tries to achieve the goal of inventing new concepts in the domain of musical instruments. It can also be customized to apply to other domains as well, such as sports. In design science the artifact is evaluated by a criteria of value or utility \parencite{march1995design}. Since our artifact can be used in different domains, the criterias and evaluations may differ from domain to domain. \citet{march1995design} state that it is important to anticipate the possible unwanted side-effects of the artifact. We want to avoid that our creative method produces results that look like creativity, but is made on a falsely grounded calculation. For example if the wrong synset were selected, a chain reaction of choices made on the wrong assumptions could happen leading to a more random result that could look like creativity by accident. The generic space could  have categories that were not really a common thing between the original concepts, forming a set of strange blending suggestions where calculations become wrong.

In our research we follow the design science activities of the research framework of \citet{march1995design} (see Figure~\ref{fig:design-science-figur}).
\citet{march1995design} divides the types of research artifacts into constructs, model, method and instantiation:
\begin{itemize}
\item \emph{Constructs} are domain-specific vocabulary. For example graph theory consist of constructs such as nodes and edges.
\item \emph{Models} describe the relationships between constructs. For example a directed acyclic graph, which has the constructs nodes and directed edges, describe the relationship involving edges only going from nodes to other nodes in the forward direction.
\item \emph{Methods} consist of steps which are used to perform a task. Methods can be based on constructs and models or used to translate from one model to another.
\item An \emph{instantiation} is an artifact that can be used in an environment. They use constructs, models and methods to form a system that demonstrates the effectiveness of their artifacts.
\end{itemize}

In our research we have built and evaluated a method for concept blending of musical instruments and other domains. Our method is based on the amalgam framework by \citet{ontanon2010amalgams}. It contains a model describing a generalization space. We apply this model with the constructs \emph{synsets} and \emph{hypernyms}, which we are able to generalize through WordNet. %This research method suits our research well since we develope

\begin{figure}
\centering
\includegraphics[width=1\linewidth]{"Figures/design science figur"}
\caption{Our research activities and outputs, following the research framework of \citet{march1995designfigure}}
\label{fig:design-science-figur}
\end{figure}


%----------------------------------------------------------------------------------------