% Chapter 1

\chapter{Introduction} % Main chapter title

\label{Chapter1} % For referencing the chapter elsewhere, use \ref{Chapter1} 

%----------------------------------------------------------------------------------------

% Define some commands to keep the formatting separated from the content 
\newcommand{\keyword}[1]{\textbf{#1}}
\newcommand{\tabhead}[1]{\textbf{#1}}
\newcommand{\code}[1]{\texttt{#1}}
\newcommand{\file}[1]{\texttt{\bfseries#1}}
\newcommand{\option}[1]{\texttt{\itshape#1}}

%----------------------------------------------------------------------------------------

\section{Motivation/Background}
Combining two concepts into a new concept that combines elements from each original concept is something humans can do rather easily. If we think of blending a horse and a bird, we can easily imagine a winged horse like the mythological Pegasus. Or when blending a car with a boat we can think of an amphibian boat. Formalizing the task of concept blending is not as straightforward. The problem consists of retrieving input data, selecting which parts of the data that should be blended and handling contradictory elements in the final blend.

For the concept blending algorithm to be useful, it is ideal that the pieces that are merging between the concepts make sense to join. In order to do this, we want to find a way to decompose the concepts into properties containing enough information so that we can look for patterns to identify suitable blending suggestions. %This step of concept blending is what we are going to focus on in this thesis.

The goal of this master thesis is to implement an algorithm to provide suggestions of elements from one concept that are suitable for transfer to another. Descriptions of the two concepts are retrieved from a public data source, DBpedia, and natural language processing tools like WordNet and Stanford Parser are used to find which elements the concept consists of. Although the concept blending algorithm may be used on any concepts, I will focus mainly on a certain domain such as musical instruments, with the goal of designing new instruments. Further the generated instruments will be evaluated on how useful they are in a musical context. The goal is to find out whether this approach can produce new instruments that make sense and find out how the algorithm can be useful to musicians and instrument makers.

%----------------------------------------------------------------------------------------

\section{Research Questions}
Here we list three questions that we want to research through incremental improvements and evaluation of the algorithm.
\begin{itemize}
\item What are the characteristics of elements in a concept that are suitable for concept blending?
\end{itemize}
The reason we want to find these are because the characteristics are what we can use as a base for the parameters in the heuristic algorithm so we can rank the good candidates for blending in any concept.
\begin{itemize}
\item How should we represent the concepts in order to find patterns of good elements for blending and detect contradictions indicating bad blends?
\end{itemize}
We want to find out which data are optimal for the input stage in order for the algorithm to work optimally and give the best suggestions. Can the concepts and their parts keep their original meaning and relations in the final results?
\begin{itemize}
\item Can a concept blending algorithm generate musical instruments that are reasonable and useful?
\end{itemize}
We want to evaluate the output of the algorithm to see if it is valuable in its environment.
%\item What is the optimal solution for preparing input data and detect contradictions, when using the concept blending algorithm to generate musical instruments?

The motivation for these questions is to uncover new knowledge on what the concept blending algorithm can be used for, how it can be used successfully and how it compares to other approaches.

%----------------------------------------------------------------------------------------