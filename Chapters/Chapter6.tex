% Chapter 6

\chapter{Results} % Main chapter title

\label{Chapter6} % For referencing the chapter elsewhere, use \ref{Chapter6} 

%----------------------------------------------------------------------------------------

\section{Identifying right criteria for ranking blends}

\subsection{Speciality}
We first started by blending musical instruments without any ranking parameters. The blending suggestions we got, were of varying quality. We recognized that the best suggestions were the ones that were most specific, such as the ones that were found due to being in the category device or artifact. Therefore we introduced two criteria to give a higher score to more specific suggestions. The first was to give a positive score based on the depth of the shared hypernym of the blending suggestion and its correlating properties in the other instrument. For example if we blend \emph{Piano} and \emph{Hurdy Gurdy}, it gave a high score to the suggestion of transfering the \emph{piano chamber} over, since the Hurdy Gurdy also has a \emph{bodily cavity} which is the hypernym of chamber. The bodily cavity had a depth score of 7, which tells us that it is far from being general.

The other criteria we introduced was a score negatively proportional to the branch length. From piano chamber, it is only one hypernym up to bodily cavity. This is good because it means that the suggestions are close to each other in speciality. Therefore the score given by the branch length criteria was only -1. The suggestion to was then given the total score of 7 minus 1, which is 6.

\begin{figure}
\centering
\includegraphics[width=0.7\linewidth]{"Figures/hurdy gurdy"}
\caption{A hurdy gurdy. Like other acoustic instruments it has a hollow cavity that enables us to hear the sound coming from the vibration of the strings}
\label{fig:hurdy-gurdy}
\end{figure}

\begin{figure}
	\centering
	\includegraphics[width=0.5\linewidth]{"Figures/Piano_HurdyGurdy"}
	\caption{One of the suggestions when blending Piano and Hurdy Gurdy. It suggests to move \emph{chamber} from Piano to Hurdy Gurdy, since it is a type of \emph{bodily cavity} which is also one of the properties of Hurdy Gurdy, thereby forming a correlation }
	\label{fig:hurdy-gurdy}
\end{figure}

\subsection{Category penalty}
We observed that a lot of suggestions were not useful so we took note of the categories bad blends shared. Different instruments were often mentioned in the abstracts, which were then interpreted as properties. It does not make sense to transfer whole instruments to other instruments. It would be too invasive since concept blending are more about blending parts of concepts, not adding whole concepts to other concepts. Therefore we introduced a penalty score for the category \emph{instruments}. Later we also decided to remove instrument properties altogether since they would still form correlations with other devices, since device is one of the hypernyms of instrument.

We also observed that properties of the category \emph{individuals} were not useful as well, since they were typically people like the inventor of the instrument or a famous musician using this instrument. It does not make sense to blend the people involved, since they are not parts or properties of the instrument. What use would there be to transfer an inventor over to an instrument they have not invented?

We initially thought that abstract entities was a bad category to blend, since good blending examples are typically physical entities. But on further inspection, we found that some of these entities were useful, like shapes and other attributes. Instead, we penalized certain abstract entities that we were more certain being bad categories. \emph{Psychological feature} are probably not interesting if they most likely are part of something other than the instrument, since instruments does not have a psychological quality. We also penalized \emph{communication}, since we are not interested in the message, language or style that is conveyed by the instrument. This is usually a matter of choice from the musician and not the instrument. Although some instruments are typical for a certain style, we do not want to define the style of new instrument in advance of its use.

\subsection{Number of correlating properties}
We observed that in some cases the top blending suggestions got an equal score. Could we find a difference in quality between these? We introduced another parameter, the number of correlating properties in the other concept. If the other concept has a lot of properties in the same shared category as the blending suggestion, this may tell us that the property is more compatible than the one with few correlations. The property may fit with the other concept in more than one way.

\subsection{Introducing factors for each parameter}
We realized that a factor of one for each parameter may not be yield the best results. Therefore we introduced a factor for each parameter so that they could be tweaked to push blends of higher quality to the top. Then we could find an optimal range of values for pushing the best suggestions to the top.

\subsection{Other potential criteria}
We thought of other possible parameters that could be used, that we did not implement and evaluate:
\begin{itemize}
\item The branch lengths of the correlating properties could be counted or averaged to give an indication of their speciality.
\item The difference between the branch lengths of the suggested property and its correlating properties could give an indication of compatibility. For example we may not want to blend a property with a branch length of 5 when the correlating properties are very general in comparison, like if it has a length of 1. Blending \emph{game equipment} like the special chess piece \emph{rook} just because the other concept had a more general game equipment like \emph{game board}, may not be as good as if it were type of board with similar level of specificity like \emph{backgammon board}.
\end{itemize}

\section{Blending musical instruments}
When blending musical instruments with no penalty categories, we got the top blending suggestions seen in \textbf{table 6.1}. Top synsets such as \emph{five} and \emph{running} where in the category \emph{abstract entities} and on their own not very interesting in describing musical instruments. If they were to be useful, we would need to model their original relation in the context, which was \emph{five strings} and \emph{running along the soundboard}.

When blending musical instruments using the penalty score 10 for the categories abstract entity, instrument and individual, we got the top blending suggestions seen in \textbf{table 6.2}. We did not exclude synsets that were in the penalized categories. This meant that top synsets often got a good score although the synset in the other instrument that shared the same hypernym was in a penalized category.

\subsection{Blending Banjo and Melodica}

Banjo properties:

\noindent\fbox{ \parbox{\textwidth}{
		
		five membrane frame resonator called head animal skin circular forms africans animal skin
		
	} }
\\\\Melodica properties:

\noindent\fbox{ \parbox{\textwidth}{
		
		key pump keyboard top played blowing mouthpiece fits hole side pressing reed popular asia modern form italy century
		
	} }
		
\begin{figure} \centering \includegraphics[width=0.7\linewidth]{"Figures/Banjo_Melodica"} \caption{Top suggestion when blending Banjo and Melodica. It suggests to move top property \emph{five} (in red) from Banjo to Melodica } \label{ Banjo-Melodica } \end{figure}

\subsection{Blending Harp and Violin }

Harp properties:

\noindent\fbox{ \parbox{\textwidth}{
		
	stringed musical soundboard africa ages renaissance family near played burma utilized modern Latin America Near East modern era
		
} }
\\\\Violin properties:

\noindent\fbox{ \parbox{\textwidth}{
		
	family regular typically perfect commonly played bow fingers prominent classical country electric forms rock iranian sometimes called fiddle particularly irish traditional regardless italy europe stradivari guarneri amati century brescia cremona austria reputation quality sound disputed hands famous mass-produced commercial cottage saxony bohemia formerly sold sears roebuck co mass violin family
		
} }
		
\begin{figure} \centering \includegraphics[width=0.7\linewidth]{"Figures/Harp_Violin"} \caption{Top suggestion when blending Harp and Violin. It suggests to move top property \emph{cottage} (in red) from Violin to Harp } \label{ Harp-Violin } \end{figure}

\begin{table}
	\caption{Examples of best suggestions when blending different pairs of instruments. Score is based on depth of word in tree (D) minus branch length from shared hypernym (BL)}
	\label{tab:topscoringexamples}
	\centering
	\begin{tabular}{l p{30mm} p{30mm} l l}
		\toprule
		\tabhead{Instrument} & \tabhead{Top scoring branch} & \tabhead{Context} & \tabhead{D - BL} & \tabhead{Score} \\
		\midrule
		Banjo &
		integer \newline
		digit \newline
		\emph{five} & \emph{five} or six-stringed instrument with a thin membrane stretched over a frame or cavity as a resonator & 7 - 3 & 4	\\
		Melodica &
		device \newline 
		\emph{keyboard} &
		The \emph{keyboard} is usually two or three octaves long & 8 - 2 &	6
		\\
		\midrule
		Harp &
		idea \newline
		plan \newline
		plan of action \newline
		play \newline
		football play \newline
		\emph{running} \newline
		(wrong synset) & The harp is a stringed musical instrument that has a number of individual strings \emph{running} at an angle to its soundboard; the strings are plucked with the fingers & 7 - 6 & 1	\\
		 &
		construction \newline
		area \newline
		enclosure \newline
		chamber \newline
		cavity resonator \newline
		\emph{soundboard} &
		The harp is a stringed musical instrument that has a number of individual strings running at an angle to its \emph{soundboard}; the strings are plucked with the fingers & 7 - 6 &	1 \\
		Violin &
		instrumentalist \newline
		\emph{violinists} & \emph{Violinists} and collectors particularly prize the instruments made by the Stradivari & 11 - 2 & 9	\\
		\\
		\bottomrule\\
	\end{tabular}
\end{table}

\begin{table}
	\caption{Examples of best suggestions when blending different pairs of instruments. Score is based on depth of word in tree (D) minus branch length from shared hypernym (BL). Words are also penalized if they have a hypernym called \emph{abstract entity}, \emph{instrument} or \emph{individual}}
	\label{tab:topscoringexamplespenalty}
	\centering
	\begin{tabular}{l p{30mm} p{30mm} l l}
		\toprule
		\tabhead{Instrument} & \tabhead{Top scoring branch} & \tabhead{Context} & \tabhead{D - BL} & \tabhead{Score} \\
		\midrule
		Banjo &
		artefact \newline
		flat solid \newline
		\emph{membrane} & five or six-stringed instrument with a thin \emph{membrane} stretched over a frame or cavity as a resonator & 6 - 3 & 3	\\
		&
		body part \newline
		system \newline
		\emph{frame} \newline
		(wrong synset)
		& five or six-stringed instrument with a thin membrane stretched over a \emph{frame} or cavity as a resonator & 6 - 3 & 3	\\
		Melodica &
		device \newline 
		\emph{keyboard} &
		The \emph{keyboard} is usually two or three octaves long & 8 - 2 &	6
		\\
		\midrule
		Harp &
		construction \newline
		area \newline
		enclosure \newline
		chamber \newline
		cavity resonator \newline
		\emph{soundboard} & The harp is a stringed musical instrument that has a number of individual strings running at an angle to its \emph{soundboard}; the strings are plucked with the fingers & 7 - 6 & 1	\\
		Violin &
		instrumentality \newline
		implement \newline
		stick \newline
		\emph{bow} & and is most commonly played by drawing a \emph{bow} across its strings \newline & 7 - 4 & 3	\\
		&
		construction \newline
		building \newline
		house \newline
		\emph{cottage} & 
		as well as still greater numbers of mass-produced commercial "trade violins" coming from \emph{cottage} industries in places such as saxony & 7 - 4 & 3	
		\\
		\bottomrule\\
	\end{tabular}
\end{table}

%----------------------------------------------------------------------------------------

\section{Alternative blending category}
\subsection{Sports}
The algorithm was tested on the alternative blending category sports. The context words to help the scoring of synsets were:

\noindent\fbox{
	\parbox{\textwidth}{
		sport athletic ball field team goal rules foul score player referee match racket club net skates skis helmet pads bat pitch court tee green win draw loss
	}
}
\\\\The penalized category was 'athletic game', as we would not want sports to be suggested, since it is the category itself.

\subsection{Blending Soccer and Golf }
Soccer properties:

\noindent\fbox{ \parbox{\textwidth}{
	
	association football commonly soccer team played two eleven players spherical ball million dependencies world game field goal object score getting opposing allowed hands arms team sport extra time World Cups
	
} }
\\\\Golf properties:
	
\noindent\fbox{ \parbox{\textwidth}{
	
	golf club ball players various hit series holes course strokes unlike games standardized playing terrains key level played arranged progression playing area putting green sand traps
			
} }
		
\begin{figure} \centering \includegraphics[width=0.7\linewidth]{"Figures/SoccerToGolf"} \caption{Top suggestion when blending Soccer and Golf. It suggests to move top property \emph{goal} (in red) from Soccer to Golf } \label{ SoccerToGolf } \end{figure}
		
\begin{figure} \centering \includegraphics[width=0.7\linewidth]{"Figures/GolfToSoccer"} \caption{Top suggestion for moving something from Golf to Soccer. It suggests to move the property \emph{terrains} (in red) from Golf to Soccer } \label{ GolfToSoccer } \end{figure}
\\\\When performing the algorithm on \emph{soccer} and \emph{golf}, it suggested to move \emph{goal} from \emph{soccer} to \emph{golf}, with a depth of 8 minus branch length 1, giving a score of 7.

%When performing the algorithm on \emph{soccer} and \emph{golf}, it suggested to move \emph{eleven} (squad) from \emph{soccer} to \emph{golf}, with a depth of 7 minus branch length 1, giving a score of 6.

It suggested to move \emph{terrains} from \emph{golf} to \emph{soccer}, with a depth of 7 minus branch length 1, giving a score of 6. The results are interesting, since it can be interpreted as suggesting golf as a team sport of eleven players. Although golf is usually played as an individual sport, there are also a number of team tournaments. Playing soccer on uneven terrain is also an interesting idea that is not common but would be possible to organize, although it may not be an evenly balanced game. One reason for why the interpretations of these suggestions worked well may be because the relations to the other elements in the concept did not contradict. \emph{Terrains} in golf matched with the part \emph{field} in soccer through their common hypernym \emph{parcel}. By not going further up the generalizing chain, the more likely relations still work.

%----------------------------------------------------------------------------------------

\section{Evaluation of blending}

%----------------------------------------------------------------------------------------