% Chapter 6

\chapter{Results} % Main chapter title

\label{Chapter6} % For referencing the chapter elsewhere, use \ref{Chapter6} 

%----------------------------------------------------------------------------------------

\section{Musical instruments}
When blending musical instruments with no penalty categories, we got the top blending suggestions seen in \textbf{table 6.1}. Top synsets such as \emph{five} and \emph{running} where in the category \emph{abstract entities} and on their own not very interesting in describing musical instruments. If they were to be useful, we would need to model their original relation in the context, which was \emph{five strings} and \emph{running along the soundboard}.

When blending musical instruments using the penalty score 10 for the categories abstract entity, instrument and individual, we got the top blending suggestions seen in \textbf{table 6.2}. We did not exclude synsets that were in the penalized categories. This meant that top synsets often got a good score although the synset in the other instrument that shared the same hypernym was in a penalized category.

\subsection{Blending Banjo and Melodica}

Banjo properties:

\noindent\fbox{ \parbox{\textwidth}{
		
		five membrane frame resonator called head animal skin circular forms africans animal skin
		
	} }
\\\\Melodica properties:

\noindent\fbox{ \parbox{\textwidth}{
		
		key pump keyboard top played blowing mouthpiece fits hole side pressing reed popular asia modern form italy century
		
	} }
		
\begin{figure} \centering \includegraphics[width=0.7\linewidth]{"Figures/Banjo_Melodica"} \caption{Top suggestion when blending Banjo and Melodica. It suggests to move top property \emph{five} (in red) from Banjo to Melodica } \label{ Banjo-Melodica } \end{figure}

\subsection{Blending Harp and Violin }

Harp properties:

\noindent\fbox{ \parbox{\textwidth}{
		
	stringed musical soundboard africa ages renaissance family near played burma utilized modern Latin America Near East modern era
		
} }
\\\\Violin properties:

\noindent\fbox{ \parbox{\textwidth}{
		
	family regular typically perfect commonly played bow fingers prominent classical country electric forms rock iranian sometimes called fiddle particularly irish traditional regardless italy europe stradivari guarneri amati century brescia cremona austria reputation quality sound disputed hands famous mass-produced commercial cottage saxony bohemia formerly sold sears roebuck co mass violin family
		
} }
		
\begin{figure} \centering \includegraphics[width=0.7\linewidth]{"Figures/Harp_Violin"} \caption{Top suggestion when blending Harp and Violin. It suggests to move top property \emph{cottage} (in red) from Violin to Harp } \label{ Harp-Violin } \end{figure}

\begin{table}
	\caption{Examples of best suggestions when blending different pairs of instruments. Score is based on depth of word in tree (D) minus branch length from shared hypernym (BL)}
	\label{tab:topscoringexamples}
	\centering
	\begin{tabular}{l p{30mm} p{30mm} l l}
		\toprule
		\tabhead{Instrument} & \tabhead{Top scoring branch} & \tabhead{Context} & \tabhead{D - BL} & \tabhead{Score} \\
		\midrule
		Banjo &
		integer \newline
		digit \newline
		\emph{five} & \emph{five} or six-stringed instrument with a thin membrane stretched over a frame or cavity as a resonator & 7 - 3 & 4	\\
		Melodica &
		device \newline 
		\emph{keyboard} &
		The \emph{keyboard} is usually two or three octaves long & 8 - 2 &	6
		\\
		\midrule
		Harp &
		idea \newline
		plan \newline
		plan of action \newline
		play \newline
		football play \newline
		\emph{running} \newline
		(wrong synset) & The harp is a stringed musical instrument that has a number of individual strings \emph{running} at an angle to its soundboard; the strings are plucked with the fingers & 7 - 6 & 1	\\
		 &
		construction \newline
		area \newline
		enclosure \newline
		chamber \newline
		cavity resonator \newline
		\emph{soundboard} &
		The harp is a stringed musical instrument that has a number of individual strings running at an angle to its \emph{soundboard}; the strings are plucked with the fingers & 7 - 6 &	1 \\
		Violin &
		instrumentalist \newline
		\emph{violinists} & \emph{Violinists} and collectors particularly prize the instruments made by the Stradivari & 11 - 2 & 9	\\
		\\
		\bottomrule\\
	\end{tabular}
\end{table}

\begin{table}
	\caption{Examples of best suggestions when blending different pairs of instruments. Score is based on depth of word in tree (D) minus branch length from shared hypernym (BL). Words are also penalized if they have a hypernym called \emph{abstract entity}, \emph{instrument} or \emph{individual}}
	\label{tab:topscoringexamplespenalty}
	\centering
	\begin{tabular}{l p{30mm} p{30mm} l l}
		\toprule
		\tabhead{Instrument} & \tabhead{Top scoring branch} & \tabhead{Context} & \tabhead{D - BL} & \tabhead{Score} \\
		\midrule
		Banjo &
		artefact \newline
		flat solid \newline
		\emph{membrane} & five or six-stringed instrument with a thin \emph{membrane} stretched over a frame or cavity as a resonator & 6 - 3 & 3	\\
		&
		body part \newline
		system \newline
		\emph{frame} \newline
		(wrong synset)
		& five or six-stringed instrument with a thin membrane stretched over a \emph{frame} or cavity as a resonator & 6 - 3 & 3	\\
		Melodica &
		device \newline 
		\emph{keyboard} &
		The \emph{keyboard} is usually two or three octaves long & 8 - 2 &	6
		\\
		\midrule
		Harp &
		construction \newline
		area \newline
		enclosure \newline
		chamber \newline
		cavity resonator \newline
		\emph{soundboard} & The harp is a stringed musical instrument that has a number of individual strings running at an angle to its \emph{soundboard}; the strings are plucked with the fingers & 7 - 6 & 1	\\
		Violin &
		instrumentality \newline
		implement \newline
		stick \newline
		\emph{bow} & and is most commonly played by drawing a \emph{bow} across its strings \newline & 7 - 4 & 3	\\
		&
		construction \newline
		building \newline
		house \newline
		\emph{cottage} & 
		as well as still greater numbers of mass-produced commercial "trade violins" coming from \emph{cottage} industries in places such as saxony & 7 - 4 & 3	
		\\
		\bottomrule\\
	\end{tabular}
\end{table}

%----------------------------------------------------------------------------------------

\section{Alternative blending category}
\subsection{Sports}
The algorithm was tested on the alternative blending category sports. The context words to help the scoring of synsets were:

\noindent\fbox{
	\parbox{\textwidth}{
		sport athletic ball field team goal rules foul score player referee match racket club net skates skis helmet pads bat pitch court tee green win draw loss
	}
}
The penalized category was 'athletic game', as we would not want sports to be suggested, since it is the category itself.

When performing the algorithm on \emph{soccer} and \emph{golf}, it suggested to move \emph{eleven} (squad) from \emph{soccer} to \emph{golf}, with a depth of 7 minus branch length 1, giving a score of 6. It suggested to move \emph{terrains} from \emph{golf} to \emph{soccer}, with a depth of 7 minus branch length 1, giving a score of 6. The results are interesting, since it can be interpreted as suggesting golf as a team sport of eleven players. Although golf is usually played as an individual sport, there are also a number of team tournaments. Playing soccer on uneven terrain is also an interesting idea that is not common but would be possible to organize, although it may not be an evenly balanced game. One reason for why the interpretations of these suggestions worked well may be because the relations to the other elements in the concept did not contradict. \emph{Terrains} in golf matched with the part \emph{field} in soccer through their common hypernym \emph{parcel}. By not going further up the generalizing chain, the more likely relations still work.

%----------------------------------------------------------------------------------------

\section{Evaluation of blending}

%----------------------------------------------------------------------------------------