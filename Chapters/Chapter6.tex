% Chapter 6

\chapter{Results} % Main chapter title

\label{Chapter6} % For referencing the chapter elsewhere, use \ref{Chapter6} 

%----------------------------------------------------------------------------------------

\section{Identifying right criteria for ranking blends}

\subsection{Speciality}
We first started by blending musical instruments without any ranking parameters. The blending suggestions we got, were of varying quality. We recognized that the best suggestions were the ones that were most specific, such as the ones that were found due to being in the category device or artifact. Therefore we introduced two criteria to give a higher score to more specific suggestions. The first was to give a positive score based on the depth of the shared hypernym of the blending suggestion and its correlating properties in the other instrument. For example if we blend \emph{Piano} and \emph{Hurdy Gurdy} (seen in Figure~\ref{fig:hurdy-gurdy}), it gave a high score to the suggestion of transferring the \emph{piano chamber} over, since the Hurdy Gurdy also has a \emph{bodily cavity} which is the hypernym of chamber. The bodily cavity had a depth score of 7, which tells us that it is far from being general.

The other criteria we introduced was a score negatively proportional to the branch length. From piano chamber, it is only one hypernym up to bodily cavity. This is good because it means that the suggestions are close to each other in speciality. Therefore the score given by the branch length criteria was only -1. The suggestion to was then given the total score of 7 minus 1, which is 6. The properties used in the calculation is visualized in Figure~\ref{fig:piano-hurdy-gurdy}.

\begin{figure}
\centering
\includegraphics[width=0.7\linewidth]{"Figures/hurdy gurdy"}
\caption{A hurdy gurdy. Like other acoustic instruments it has a hollow cavity that enables us to hear the sound coming from the vibration of the strings}
\label{fig:hurdy-gurdy}
\end{figure}

\begin{figure}
	\centering
	\includegraphics[width=0.5\linewidth]{"Figures/Piano_HurdyGurdy"}
	\caption{One of the suggestions when blending Piano and Hurdy Gurdy. It suggests to move \emph{chamber} from Piano to Hurdy Gurdy, since it is a type of \emph{bodily cavity} which is also one of the properties of Hurdy Gurdy, thereby forming a correlation }
	\label{fig:piano-hurdy-gurdy}
\end{figure}

\subsection{Category penalty}
We observed that a lot of suggestions were not useful so we took note of the categories bad blends shared. Different instruments were often mentioned in the abstracts, which were then interpreted as properties. It does not make sense to transfer whole instruments to other instruments. It would be too invasive since concept blending are more about blending parts of concepts, not adding whole concepts to other concepts. Therefore we introduced a penalty score for the category \emph{instruments}. Later we also decided to remove properties of the category instruments altogether since they would still form correlations with other devices, since device is one of the hypernyms of instrument.

We also observed that properties of the category \emph{individuals} were not useful as well, since they were typically people like the inventor of the instrument or a famous musician using this instrument. It does not make sense to blend the people involved, since they are not parts or properties of the instrument. What use would there be to transfer an inventor over to an instrument they have not invented?

We initially thought that abstract entities was a bad category to blend, since good blending examples are typically physical entities. But on further inspection, we found that some of these entities were useful, like shapes and other attributes. Instead, we penalized certain abstract entities that we were more certain being bad categories. \emph{Psychological feature} are probably not interesting if they most likely are part of something other than the instrument, since instruments does not have a psychological quality. We also penalized \emph{communication}, since we are not interested in the message, language or style that is conveyed by the instrument. This is usually a matter of choice from the musician and not the instrument. Although some instruments are typical for a certain style, we do not want to define the style of new instrument in advance of its use. \emph{Body parts} are not something we want to blend as parts, but in future approaches that look for relations, this would be useful to describe relations like \emph{plucked by fingers}.

\subsection{Number of correlating properties}
We observed that in some cases the top blending suggestions got an equal score. Could we find a difference in quality between these? We introduced another parameter, the number of correlating properties in the other concept. If the other concept has a lot of properties in the same shared category as the blending suggestion, this may tell us that the property is more compatible than the one with few correlations. The property may fit with the other concept in more than one way.

\subsection{Introducing factors for each parameter}
We realized that a factor of one for each parameter may not be yield the best results. Therefore we introduced a factor for each parameter so that they could be tweaked to push blends of higher quality to the top. Then we could find an optimal range of values for pushing the best suggestions to the top.

\subsection{Other potential criteria}
We thought of other possible parameters that could be used, that we did not implement and evaluate:
\begin{itemize}
\item The branch lengths of the correlating properties could be counted or averaged to give an indication of their speciality.
\item The difference between the branch lengths of the suggested property and its correlating properties could give an indication of compatibility. For example we may not want to blend a property with a branch length of 5 when the correlating properties are very general in comparison, like if it has a length of 1. Blending \emph{game equipment} like the special chess piece \emph{rook} just because the other concept had a more general game equipment like \emph{game board}, may not be as good as if it were type of board with similar level of specificity like \emph{backgammon board}.
\end{itemize}

\section{Blending musical instruments}

Here are a couple of blending examples when using the application to blend musical instruments. First we present the blending suggestions when we use an unaltered abstract as it was retrieved directly from Wikipedia. Then we try again, but this time using an improved abstract we created manually by removing unhelpful sentences and adding new useful facts. If synsets were wrong, we also manually corrected them. This way we can evaluate the second and third step of the application in an optimal environment.

\subsection{Blending Banjo and Melodica}

\subsubsection{Unaltered abstract}

Banjo properties:

\noindent\fbox{ \parbox{\textwidth}{
		
		five membrane frame resonator called head animal skin circular forms africans animal skin
		
	} }
\\\\Melodica properties:

\noindent\fbox{ \parbox{\textwidth}{
		
		key pump keyboard top played blowing mouthpiece fits hole side pressing reed popular asia modern form italy century
		
	} }
\\\\Top blending suggestion was to move the word \emph{five} from banjo to melodica with the score of 4, correlating with the property \emph{century} through the common category \emph{integer}. \emph{Century} is a part of the cultural description of melodica, rather than the more useful sentences describing physical attributes.

\subsubsection{Optimal environment}
By improving the banjo abstract we got more relevant properties like \emph{strings} and \emph{frets}, and a variety of materials which the banjo parts and strings are made of.
\\Improved banjo properties:\\
\noindent\fbox{ \parbox{\textwidth}{
		
	four five six stringed thin membrane stretched frame cavity resonator called head plastic animal skin circular body consists rim wood metal tensioned similar tone ring assembly project sound tuned friction tuning peg gear worm machine frets standard played strung strings string wound steel alloy nylon gut achieve old-time separate plate pot forward volume
		
} }	
\\\\In the new set of melodica properties we got rid of cultural words like \emph{Asia}, \emph{Italy} and \emph{century}, and retrieved more relevant ones like the materials \emph{plastic} and \emph{wood}.
\\Improved melodica properties:\\
\noindent\fbox{ \parbox{\textwidth}{
			
	keyboard top played blowing air mouthpiece fits hole pressing key flow reed two three light portable majority plastic wood
			
} }
\\\\The top blending suggestion we got was introducing \emph{machine} as in the guitar's worm gear machine head for tuning the strings (see Figure~\ref{fig:worm-gear}). It correlated with the \emph{reed} of melodica, through the category \emph{mechanical device}. It is more specific than the other categories, but still so general that they are not closely related in function. Although we found the noun phrase \emph{machine head}, it did not exist in WordNet, giving us a indication of the limits of its database. The user may be inspired to create a melodica where you can tune the reeds with a similar tuning device as guitar.
The calculation of this blending suggestion is visualized in Figure~\ref{fig:Banjo-Melodica2}.

\begin{figure}
\centering
\includegraphics[width=0.5\linewidth]{"Figures/worm gear"}
\caption{A worm gear which is used in a machine head for tuning the strings of banjos and guitars}
\label{fig:worm-gear}
\end{figure}

\begin{figure} \centering \includegraphics[width=0.7\linewidth]{"Figures/Banjo_Melodica2"} \caption{Top suggestion when blending Banjo and Melodica using improved abstracts. It suggests to move top property \emph{machine} (in red) as in \emph{worm gear machine head} from Banjo to Melodica } \label{fig:Banjo-Melodica2} \end{figure}
		

\subsection{Blending Harp and Violin }

\subsubsection{Unaltered abstract}

Harp properties:

\noindent\fbox{ \parbox{\textwidth}{
		
	stringed musical soundboard africa ages renaissance family near played burma utilized modern Latin America Near East modern era
		
} }
\\\\Violin properties:

\noindent\fbox{ \parbox{\textwidth}{
		
	family regular typically perfect commonly played bow fingers prominent classical country electric forms rock iranian sometimes called fiddle particularly irish traditional regardless italy europe stradivari guarneri amati century brescia cremona austria reputation quality sound disputed hands famous mass-produced commercial cottage saxony bohemia formerly sold sears roebuck co mass violin family
		
} }

\begin{figure} \centering \includegraphics[width=0.7\linewidth]{"Figures/Harp_Violin"} \caption{Top suggestion when blending Harp and Violin. It suggests to move top property \emph{cottage} (in red) from Violin to Harp } \label{fig:Harp-Violin} \end{figure}

\subsubsection{Optimal environment}
Harp properties:\\
\noindent\fbox{ \parbox{\textwidth}{
		
	stringed strings angle soundboard plucked size played larger heavy rest floor catgut nylon metal combination neck resonator frame pillar support arch bow modern extend range adjusting levers pedals modify pitch
		
} }
\\\\Violin properties:\\
\noindent\fbox{ \parbox{\textwidth}{
			
	wooden string family scroll neck fingerboard bridge sound hole tailpiece four strings tuned commonly played drawing bow plucking called wood strung gut synthetic steel sound hole
			
} }
\\\\We got two top blending suggestions with the score 5. The first was to blend the violin \emph{bow} with harp, correlating through the category \emph{implement} with the harp \emph{pedals} that bend the strings. Harpists have played harps with bows, but since you have to insert the bow between harp strings, it is not easy to change notes. However since the harpists have two available hands, they can use two bows. The calculation is visualized in Figure~\ref{fig:Harp-ViolinOptimal}.

The other top suggestion was to move the \emph{bridge} from violin to harp, correlating with \emph{levers} through the category \emph{device}. Not a very interesting addition, since harp strings is mounted in a way that transmit enough vibration to the soundboard already.

The top suggestion to move from harp to violin was \emph{pedals} which modify pitch, correlating with the violin \emph{bow} through the category \emph{implement}. This could be an interesting feature, but since the violin has no frets, modifying pitch is already possible with the glissando technique where you slide fingers up or down the string. However if you want to play lower than the open string or bend further than the fingers can reach (unlikely on a small violin), a pedal could be useful.
		
\begin{figure} \centering \includegraphics[width=0.7\linewidth]{"Figures/Harp_ViolinOptimal"} \caption{Top suggestion when blending Harp and Violin. It suggests to move top property \emph{bow} (in red) from Violin to Harp } \label{fig:Harp-ViolinOptimal} \end{figure}


\subsection{Blending Cello and Oboe}

\subsubsection{Unaltered abstract}

Cello properties:

\noindent\fbox{ \parbox{\textwidth}{
		
	bowed sometimes plucked string four strings tuned low lower double rests floor support weight carved wood violin family
		
} }
\\\\Oboe properties:

\begin{figure} \centering \includegraphics[width=0.7\linewidth]{"Figures/Cello_Oboe"} \caption{Top suggestion when blending Cello and Oboe. It suggests to move top property (in red) from Oboe to Cello } \label{Cello-Oboe} \end{figure}

\noindent\fbox{ \parbox{\textwidth}{
			
	double reed wood soprano cm metal flared bell sound blowing column tone
			
} }	
\\\\This blend was, as indicated by its low score, quite general (see Figure~\ref{Cello-Oboe}). The blending suggestion could have been any artifact on the oboe. It is not easy to imagine how a flared bell could fit on a cello without further elaboration. Perhaps the body could be given this shape and resonate in an interesting way. But since air is not blown through the body of a cello like a wind instrument, it might not have the same effect on the timbre as on an oboe. But there exists a type of drums called Staccato drums which have a notched horn shape to change its sound quality and volume.

\subsubsection{Optimal environment}

\subsection{Blending Dobro and Harp }
\subsubsection{Unaltered abstract}

Dobro properties:\\
\noindent\fbox{ \parbox{\textwidth}{
		
		word popular term resonator brand currently gibson inverted biscuit national steel solid body
		
} }
\\\\Harp properties:\\
\noindent\fbox{ \parbox{\textwidth}{
			
	stringed soundboard plucked africa popularity ages renaissance near played myanmar utilized modern era Latin America Near East modern era
			
} }
\\\\We started with the unaltered abstract.
With these properties we got similar blending suggestions from both Dobro and Harp. It suggested to move \emph{body}, the cavity resonator, from dobro to the harp, while it also suggested to move the soundboard, which is also a cavity resonator, from harp to dobro. Being both types of cavity resonator, they got the same score. Although a dobro has a body, this word came from a reference to a solid body guitar, which is not the same. Other suggestions were not nearly as good in terms of score, moving steel from dobro to harp got as score of 1, but it was the wrong synset. Not the material but a sharpener. Manually improving the WSD did not improve the suggestions and scores significantly. 
		
\begin{figure} \centering \includegraphics[width=0.7\linewidth]{"Figures/Dobro_Harp"} \caption{Top suggestion when blending Dobro and Harp. It suggests to move top property body (in red) from Dobro to Harp} \label{Dobro-Harp} \end{figure}

\subsubsection{Optimal environment}

After improving the abstracts by manually selecting useful parts from the Wikipedia abstract, we got a somewhat different result. Body was no longer a property, but soundboard from harp was still the best suggestion since dobro had a cavity resonator, which is its hypernym. The best suggestion of dobro was aluminium from its cast aluminium spider, linking with the metal that strings may be made of in a harp. Other suggestions that were less rewarded, but more interesting would have been the cone or bowl forming the resonator. An interesting but low scoring suggestion from harp is its pitch bending levers or pedals.


\subsection{Blending Guitar and Violin}
\subsubsection{Unaltered abstract}

\subsubsection{Optimal environment}


%----------------------------------------------------------------------------------------

\section{Alternative blending category}
\subsection{Sports}
The algorithm was tested on the alternative blending category sports. The context words to help the scoring of synsets were:

\noindent\fbox{
	\parbox{\textwidth}{
		sport athletic ball field team goal rules foul score player referee match racket club net skates skis helmet pads bat pitch court tee green win draw loss
	}
}
\\\\The penalized category was 'athletic game', as we would not want sports to be suggested, since it is the category itself.

\subsection{Blending Soccer and Golf }

\subsubsection{Unaltered abstract}

Soccer properties:

\noindent\fbox{ \parbox{\textwidth}{
	
	association football commonly soccer team played two eleven players spherical ball million dependencies world game field goal object score getting opposing allowed hands arms team sport extra time World Cups
	
} }
\\\\Golf properties:
	
\noindent\fbox{ \parbox{\textwidth}{
	
	golf club ball players various hit series holes course strokes unlike games standardized playing terrains key level played arranged progression playing area putting green sand traps
			
} }
\\\\When performing the algorithm on \emph{soccer} and \emph{golf}, it suggested to move \emph{goal} from \emph{soccer} to \emph{golf}, with a depth of 8 minus branch length 1, giving a score of 7 (see Figure~\ref{ SoccerToGolf }).


\begin{figure} \centering \includegraphics[width=1\linewidth]{"Figures/SoccerTerrain"} \caption{An interpretation of how the property \emph{terrain} might be used in the game of soccer} \label{fig:SoccerTerrain} \end{figure}

\begin{figure} \centering \includegraphics[width=0.5\linewidth]{"Figures/SoccerToGolf"} \caption{Top suggestion when blending Soccer and Golf. It suggests to move top property \emph{goal} (in red) from Soccer to Golf } \label{fig:SoccerToGolf} \end{figure}

\begin{figure} \centering \includegraphics[width=0.5\linewidth]{"Figures/GolfToSoccer"} \caption{Top suggestion for moving something from Golf to Soccer. It suggests to move the property \emph{terrains} (in red) from Golf to Soccer } \label{fig:GolfToSoccer} \end{figure}

\subsubsection{Optimal environment}

%----------------------------------------------------------------------------------------


%----------------------------------------------------------------------------------------