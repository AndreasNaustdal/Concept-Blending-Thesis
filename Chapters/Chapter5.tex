% Chapter 5

\chapter{Concept Blending Algorithm} % Main chapter title

\label{Chapter5} % For referencing the chapter elsewhere, use \ref{Chapter5} 

%----------------------------------------------------------------------------------------

\section{Concept blending}

%----------------------------------------------------------------------------------------

\section{Identifying noun phrases}
When analysing a text for useful properties, some of them may be described by multiple words or a noun phrase. Take for example the two-worded property sustaining pedal, which is one of the pedals on a piano that lifts the dampers from the strings to let them continue vibrating. The word sustaining and pedal can not sufficiently describe the sustaining pedal on their own. The word sustaining depends on the word pedal, and it may not be clear if the word pedal is a sustaining pedal, a soft pedal or any other pedal unrelated to pianos. Therefore identifying noun phrases can lead to finding more useful properties in the text.

%----------------------------------------------------------------------------------------

\section{Word sense disambiguation in descriptions}
Word sense disambiguation

%----------------------------------------------------------------------------------------

\section{Creative strategy}

%----------------------------------------------------------------------------------------

\section{Evaluation of blending}

%----------------------------------------------------------------------------------------