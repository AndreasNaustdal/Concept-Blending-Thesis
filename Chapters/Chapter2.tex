% Chapter 2

\chapter{Literature review} % Main chapter title

\label{Chapter2} % For referencing the chapter elsewhere, use \ref{Chapter2} 

%----------------------------------------------------------------------------------------

\section{Creativity}
%When looking at different definitions of creativity, it is quite abstract. Here are some definitions of creativity:

%Wikipedia:
%Creativity is a phenomenon whereby something new and somehow valuable is formed. The created item may be intangible (such as an idea, a scientific theory, a musical composition, or a joke) or a physical object (such as an invention, a literary work, or a painting).

%Google/Oxford:
%the use of imagination or original ideas to create something; inventiveness.
%"firms are keen to encourage creativity"
%synonyms:
%inventiveness, imagination, imaginativeness, innovation, innovativeness, originality, individuality; 
%artistry, expressiveness, inspiration, vision, creative power, creative talent, creative gift, creative skill, resourcefulness, ingenuity, enterprise
%"challenging objectives motivate staff and encourage creativity"


Cambridge Dictionary gives the following definition of creativity:
the ability to produce original and unusual ideas, or to make something new or imaginative.

%Dictionary.com
%noun
%1.
%the state or quality of being creative.
%2.
%the ability to transcend traditional ideas, rules, patterns, relationships, or the like, and to create meaningful new ideas, forms, methods, interpretations, etc.; originality, progressiveness, or imagination:
%the need for creativity in modern industry; creativity in the performing arts.
%3.
%the process by which one utilizes creative ability:
%Extensive reading stimulated his creativity.

%Merriam-Webster:
%Definition of creativity
%1:  the quality of being creative
%2:  the ability to create her artistic creativity

Creativity happens in different ways. Sometimes a new invention is found because the right elements were available, and a combination of these created something new. For instance the first electric light was invented when Humphry Davy had access to the world's largest battery at the time, and passing current through a strip of platinum produced light. In other cases creativity is goal-driven. When electric light was commercialized it went through a creative process of choosing the right materials and finding an cost-efficient production method with the goal of making the light-bulb profitable.
%It seems hard to define specifically how creativity is actually done, since it can be done in many ways. 
%The established literature seems to branch into different methods. Examples of creative tasks include invention, problem solving and art. Our approach will follow the theory of conceptual blending.

\citet{busse1980theories} lists seven categories of creativity theories: psychoanalytic, gestalt, association, perceptual, humanistic, cognitive-developmental and composite theories.
They also draw a line between convergent and divergent problems that the theories are applicable to. The difference there is convergent problems have one or few right answers while divergent has many possible solutions. One end is a more goal-driven and problem-solving type of creativity and the other end includes more open-ended activities such as art and conceiving novel ideas.

In the association theories, creativity is thought to come from associations.
Arthur Koestler combines associationism and psychoanalytic concepts in what he called the \emph{bisociation} theory \parencite{busse1980theories}.
In bisociation, two independent matrices of ideas are combined, guided by subconscious processes.
Koestler use the word \emph{matrix} to represent abilities, habits, skills or any pattern of behaviour governed by fixed rules. \parencite{koestler1964act}.
Bisociation inspired the theory of conceptual blending by Gilles Fauconnier and Mark Turner \parencite{fauconnier2002way}.

\citet{boden2004creative} lists three types of creativity: exploratory, transformational, and combinatorial. Exploratory is exploring the possibilities of a familiar domain. Transformational is changing rules and opening restrictions in order to break out of the known domain. Combinatorial creativity combines familiar concepts in a novel way, going outside known rules to create new ideas. An example of this can be found in Lego, where using a combination of bricks with different shapes and styles yield a lot more creative combinations than only using the standard 2x2 brick. \parencite{brainpickings} Conceptual blending has been used in efforts to create computational approaches to combinatorial creativity.

%----------------------------------------------------------------------------------------

%\section{Computational creativity}


%----------------------------------------------------------------------------------------

\section{Conceptual blending}
Conceptual blending is a theory proposed by Gilles Fauconnier and Mark Turner. It works by projecting structure from input mental spaces to a separate, "blended" mental space. \parencite{fauconnier1998conceptual}
The mental spaces contain structured elements and are interconnected to model dynamic mappings in thought and language.
Conceptual blending can be explained using a model of four mental spaces consisting of two input spaces, the generic space and blend space (see Figure~\ref{fig:concept-blending-firkant}).
Each input space is a partial model of the corresponding concept.
The generic space contains what the input spaces have in common.
Elements from the generic space are mapped onto the blend space, and may be blended with elements from the input spaces that are not mapped to the generic space.
%According to \parencite{reference3}: “concept blending constitutes a cognitive process which allows for the combination of certain elements (and their relations) from originally distinct conceptual spaces into a new unified space combining these previously separate elements and allowing the performance of reasoning and inference over the combination.”
\begin{figure}
\centering
\includegraphics[width=0.7\linewidth]{"Figures/concept blending firkant"}
\caption{Conceptual blending model showing mental spaces (circles) and mappings of counterpart connections (solid lines) \parencite{fauconnier1998conceptualfigure}}
\label{fig:concept-blending-firkant}
\end{figure}
%----------------------------------------------------------------------------------------

\section{Computational approaches}
Efforts have been made to create a computational algorithm using the conceptual blending theory.
\citet{li2012goal} presents an algorithm with an emphasis on context-induced goals to prune the search space. %They also distinguish between two different artifacts of conceptual blending, semiotic expressions and standalone concepts.
\citet{martinez2011towards} uses a logic-based approach not only in a mathematical setting but also in disparate problems such as rationality puzzles and noun-noun combinations.
\citet{besold2015generalize} uses a similar logic-based approach but using amalgams developed by \citet{ontanon2010amalgams}.


%----------------------------------------------------------------------------------------

%\section{Analogy}

%Relasjonar er viktig i analogiar
%Analogiar er viktig for concept blending
%Generalisering?

%Dedre Gentner 1983

%Structure mapping engine is a direct mapping approach

%Generalization based models of analogy
%Smaling 2003
%%HDTP (Schmidt et al. 2014)

%Min løysing er delvis inspirert av analogi ved generalisering?

%----------------------------------------------------------------------------------------

\subsection{Amalgams}
One approach to conceptual blending is the use of amalgams \parencite{ontanon2010amalgams}.
An amalgam uses generalization of two concepts to make them compatible for blending through unification of the two generalizations. When parts of the concepts share a similar generalization, but are different, they are not unifiable. But they can unify with the \emph{least general generalization}. To make the concepts unifiable, we can therefore generalize corresponding pairs of parts between the concepts. For instance, when creating an amalgam of a \emph{red French vehicle} and a \emph{German minivan}, we can make them unifiable by generalizing the first to a \emph{red European vehicle}. When unifying we get a \emph{red German minivan}. Or if we generalize the second to a \emph{European minivan}, we can unify to a \emph{red French minivan}.
%An amalgam of two descriptions is a new description that contains parts from these two descriptions. (sitat frå Besold eller Ontanón and Plaza 2010?)

%Subsumption relasjon (hyponym subsuming hypernym)
%All informasjon i hypernymet er sant for hyponymet

%----------------------------------------------------------------------------------------

\section{Meaning and context}
An important part of concept blending is not only representing the parts to blend as words, but also their meaning which depends on the context. \parencite{Reference6} propose that there is a difference between lexical concepts and meaning. Meaning is not a function of language, but arise from the use of language. \parencite{Reference6} provides a theory of lexical concept integration, the Theory of Lexical Concepts and Cognitive Models (LCCM). 

%----------------------------------------------------------------------------------------

%\section{The COINVENT Project}

%----------------------------

\section{Natural Language Processing}
Natural language processing (NLP) is a field in computer science where computers interact with and process human natural language.
We use NLP tools such as WordNet and Stanford coreNLP to process the concept descriptions into input spaces for blending.

\subsection{Machine-readable dictionary}
A machine-readable dictionary (MRD) is a lexical database that can queried by a computer program.
The quality of the representation of concepts depend on the number of words in the database.
The more words there are in the domain of musical instruments, the more information we can get from the WikiPedia descriptions.
We use the WordNet database, as it was the only MRD we found that was large enough for our application. \parencite{fellbaum1998wordnet}

\subsection{Synsets}
Words that mean the same thing are called synonyms. A synset is a set of synonyms. An ambiguous word can have many different synsets. A \emph{crane} is not only a bird, it is a device that lifts and objects. The most extreme case is the word \emph{break} which can mean 75 different things.

\subsection{Hypernyms}
Words belong in categories. A rook belongs to the category chess piece. Chess piece belongs to the category of pieces of any type of board game. If we continue generalizing we get the categories game equipment, equipment, instrumentality, artifact, unit, physical object and physical entity until we finally reach entity. The category something belongs to is called a \emph{hypernym}, and a specific instance of this is called a \emph{hyponym}. A hypernym is the hypernym of a hyponym.

\subsection{Part-of-speech}
Words can be categorized based on grammatical properties. These categories are called part-of-speech and include nouns, verbs, adjectives, adverbs, pronouns, prepositions, conjunctions, interjections, numerals, articles and determiners.

\subsection{Word sense disambiguation}
A word can have different meanings called \emph{word senses}. \emph{Word sense disambiguation} (WSD) is the problem of identifying which word sense is the correct when a word can mean different things in a sentence or text. Approaches are commonly divided into supervised, unsupervised and knowledge-based WSD \parencite{navigli2009word}.

Supervised methods create a classifier by training on data sets of texts where the words have been labeled with the correct sense. \parencite{navigli2009word}.

Unsupervised methods are based on data sets of unlabeled texts. \parencite{navigli2009word}. %GRATIS Skrive meir om korleis dei er avhengig av naboorda?

The \emph{Lesk algorithm} is an knowledge-based method where a machine readable dictionary is used to look up the senses and definitions of each word. The definitions which have the most overlapping words with definitions of nearby words are chosen to be the correct definition. \parencite{lesk1986automatic} A variant of this method is called \emph{Simple Lesk} \parencite{kilgarriff2000english}. It differs from the original by returning the definition which has the most overlapping words with the surrounding sentences. 
The \emph{Lesk algorithm} is a knowledge-based method where a machine readable dictionary is used to look up the senses and definitions of each word. The definitions which have the most overlapping words with definitions of nearby words are chosen to be the correct definition. \parencite{lesk1986automatic} A variant of this method is called \emph{Simple Lesk} \parencite{kilgarriff2000english}. It differs from the original by returning the definition which has the most overlapping words with the surrounding sentences. 
Since we are using WikiPedia articles, an approach for extending this technique could be to compare versions of the article in different languages, and look for overlapping senses in these versions by using multilingual versions of WordNet. For each word in english, look for the sense that has the highest frequency in the versions of the other languages.

\subsection{Stop-words}
Stop-words is a concept where unwanted, regular appearing words are removed from the sample text. Words such as \emph{I}, \emph{is}, \emph{do} or other common binding words are discarded and replaced by empty strings. By removing these stop-words the only words remaining are words that are more likely to be significant. 