% Chapter 2

\chapter{Methods and methodologies} % Main chapter title

\label{Chapter2} % For referencing the chapter elsewhere, use \ref{Chapter2} 

%----------------------------------------------------------------------------------------

\section{Personal Kanban}
In order to keep my work organized and productive, I will use the agile method called Personal Kanban. To visualize the workload, Personal Kanban uses a value stream which are columns that divide the work based on the progress of each task. I will use the simplest one that divides the work into three columns. The first is the project backlog which includes all the tasks that awaits production. The middle column is the tasks currently being worked on, which should be a limited number of tasks so that tasks are not getting stuck half-finished and distracting the task one is actually doing. The last column is the tasks that have been completely done so that there are no reason to be distracted by it anymore. \parencite{Reference1}
\\\\
To keep a workflow where I can spend my energy on making sure the model, view and controller pattern is interacting well together, I will limit my task number to 3. This way I can select one task from each pattern that are depending to each other, while unfinished tasks should not be slowing down the workflow.
To review my progress, I will have a daily stand-up meeting similar to the kind used in XP \parencite{Reference2} where I will answer the following questions:
\begin{enumerate}
\item What was accomplished yesterday?
\item What will be attempted today?
\item What problems are causing delays?
\end{enumerate}


%----------------------------------------------------------------------------------------

\section{Proof of concept}

%----------------------------------------------------------------------------------------