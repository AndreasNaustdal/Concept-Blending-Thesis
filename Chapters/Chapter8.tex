% Chapter 8

\chapter{Discussion} % Main chapter title

\label{Chapter8} % For referencing the chapter elsewhere, use \ref{Chapter8} 

%----------------------------------------------------------------------------------------

\section{Aspects with approach/algorithm}

\subsection{Properties from WikiPedia abstract}
Sometimes highly relevant properties are not in the WikiPedia abstract. Sometimes they are there but as a verb instead of noun. For example cello has the adjective \emph{bowed}, but it is far more relevant to find the noun \emph{bow}, which is not found. A solution could be to find all nouns by nominalization of words from other part-of-speech. WordNet seems to have the capability of returning some derivationally related forms of a word, but this was not included in the JavaScript library we used. Finding these computationally is not a straightforward process in itself but could be done using a heuristic method where you remove or change suffixes of the word. With the word \emph{bowed} you can search for the suffix \emph{-ed} and remove this, but this method also turns words like \emph{tuned} into totally different words like \emph{tun}, which is a physical object we do not want to correlate with others. Therefore we would either need a more advanced search technique. We would just discard any word that does not return synsets in WordNet, but the more words to search, the longer it will take to finish the search.

It may be possible to detect the quality of a sentence for finding properties. Sometimes an article try to explain a concept by what it is not. For example if it includes phrases like \emph{rather than} or \emph{instead of}, it may suggest that the words afterwards are properties not typical of the original concept.
We would rather select from the properties of sentences that describe very exact how the concept is, and not pick up properties that are opposite scenarios. By looking for words that are characteristic of deviations, it could be possible to reduce unwanted properties by removing the sentences they came from.
Some words can indicate more cultural descriptions of instruments, rather than physical descriptions. These include geographical words like countries and people, styles, years etc... These are typically \emph{proper nouns}.
Therefore a measure of quality could be the number of \emph{proper nouns}. These words are often used when describing cultural matters and historical events and not parts and properties.

Neighbouring sentences often share the same context, so a sentence between two bad ones may also be disfavoured.

Some words have multiple synsets that are useful. For example the word \emph{brass} in \emph{brass instrument} has a synset for being a part of the brass instrument family, and also for being a material. Although the brass family synset is the correct one and also the one chosen in our program, the material is actually more useful when blending a brass instrument. The material is interesting because it is a part of the physical build of the instrument and something that can be transfered to other instruments. This move changes the other instruments sound and is therefore a more interesting creative choice, whereas brass family is just a categorization that is not really useful. One way to solve this could be to use all synsets that has a score over a certain threshold, as individual properties.

%----------------------------------------------------------------------------------------

\subsection{Data structure}
The data structure I have used for the concepts in my solution is an array of synsets extracted from a WikiPedia article. These synsets are intended to work as having the relation of being a part of the concept. However sometimes the word has actually a different relation to the concept than being a part of it. For example in the concept saxophone, we have the synset woodwind. In this case the relation is that the saxophone is in the woodwind family. In fact of the 28 extracted synsets, only the synset mouthpiece made any sense as being a part of the saxophone. Although it is correct that the synset brass is a part of the saxophone, a better relation would be the brass is its material.

Following this observation, it would make sense to use a data structure of a list predicates instead of words. Blending similar predicates might give more compatible suggestions. Extracting predicates from descriptions like an WikiPedia article would be a hard task. The problem could be simplified by requiring a human to input the predicates of each concept. This would restrict the usability and it would no longer be automatic, but it could be used to test the blending suggestion system with a more optimal input data structure. Finding the right synset of a predicate word will probably be harder than finding it in descriptions of text, since there are less words to define the context. The value of using a generalization by hypernyms approach may therefore be limited.

%Bruke predikat/relasjonar som input? gjelder generaliseringsgreiene fortsatt? a saxophone is made of brass, a saxophone is made of alloy (men ikkje kva som helst) hmmm

%The data structure I have used in my solution is an array of synsets extracted from a WikiPedia article. These synsets are intended to work as a predicate CONSIST_OF, that it is true that the concept consists of the synset. However sometimes the word is actually a different predicate than CONSIST_OF. For example in the concept saxophone, we have the synset of the instrument family woodwind. The relation between 

When blending a \emph{guitar} and \emph{violin} the program suggest to move the words plectrum from \emph{guitar} and \emph{bow} from violin. These are blends that have already been used by several musicians. However the reason they form a correlation is because they are devices, and not because they share the relation of being used to vibrate strings. This means that it can form a correlation with devices that do not have this relation. In order to find this relation, the program would have to look at the sentence \emph{(...)plucking the strings (...) with a pick(...)} and find that pick is used for plucking the strings. Then pick and plucking could be generalized into the words devices and cause motion, which is also true for \emph{bow}. This indicates that using sentences or predicates extracted from sentences rather than words and noun phrases would form better correlations. The scoring of blending suggestions could then have to be based on combinations of generalization-based scores for each word in the sentence.


%----------------------------------------------------------------------------------------

\subsection{Criteria}
Sometimes a shorter branch length score may not be better. For example if the abstract has both a word such as \emph{butterfly valve} and its hypernym \emph{valve}, the more general valve will get higher score and suggested as a better blend. Both words were originally referring to butterfly valve. On the other hand, a more general suggestion leads to more room for interpretation, which could lead to other implicit specializations since there are multiple types of valves. %Men det er kanskje ein grunn til at det er bedre? Unngår veldig spesifikke utbroderingar og lar det vere opent for tolkning?

%----------------------------------------------------------------------------------------

\subsection{Penalized categories}
Which categories are the most useful for concept blending of musical instruments? We penalized \emph{instruments} because we did not want to blend whole instruments into other instruments. \emph{Individuals} were penalized because we did not want to consider people as a part of the instruments. Often these people were the inventor or famous musicians. We penalized \emph{abstract entities} because it makes most sense to blend physical parts of the instruments. However categories like \emph{shapes} and \emph{materials} are also \emph{abstract entities}, and these would be interesting to blend. Therefore it might be more useful to select the categories to include rather than penalize or exclude. The best method may be to use both. For example we include \emph{physical entities} and \emph{attributes}, but penalize/exclude \emph{instruments} and \emph{individuals}. Some categories such as \emph{amounts} may have interesting information, but only if we can find and represent the original relation it had to an object in the data structure. For example \emph{five} is not interesting to blend its own, but \emph{five strings} is.

%----------------------------------------------------------------------------------------

\subsection{Other creative strategies}
There are other relations possible to retrieve using WordNet that may be useful in blending strategies. WordNet link synsets to their \emph{antonyms} which is their opposite meaning. The antonym of a top blending suggestion could be just compatible creatively when blending. For example the antonym of \emph{bowed}, \emph{plucked}, would be useful in similar contexts as \emph{bowed} itself. \emph{Part meronyms} are parts of the corresponding word, which may be useful if they are of a different category than the word itself. Then they may form new connections in the generic space, leading to more interesting blending suggestions. We could go even further as well and find antonyms of meronyms.

%----------------------------------------------------------------------------------------

\subsection{Elaboration and interpretation}
Some kinds of blending suggestions require some kind of elaboration to be able to work in practice. For example if the suggestion is \emph{tone}, taken from the distinct tone of the original instrument, some elaborating has to be done in order for the instrument to generate the new tone in practice. Perhaps this distinct tone relies on other properties which did not join the blend. Making a cello sound like an oboe is not obvious on its own.

How can elaboration be done computationally? One way could be to look at the correlating properties, and look for existing relations there. %and see how they are related to the concept.

%----------------------------------------------------------------------------------------

\subsection{Goals}
Conceptual blending can be used for divergent problems. The point would then be to create something new and novel, but still compatible and working. The interesting factor is the open and surprising creative choices that still work because they fit in the original system of relations. The most compatible blends are ranked at the top. But conceptual blending can also be used in more converging problems, where the blending process is driven by a goal. The more specific the concepts you are looking for, the more you can prune the search space to find exactly these type of results. In this thesis we have tried use conceptual blending to invent new musical instruments. Before we pruned our search space using penalty categories, the results were very open and less related to our goal. The more carefully we pruned bad categories, the better blending results we got. Therefore it would seem that the focus of a goal-driven conceptual blending approach would be how to represent the goal and its ability to prune the search space in the right direction.

%----------------------------------------------------------------------------------------

\subsection{Meaning}
\parencite{Reference5} argues that it is unlikely that any meaning is truly context-invariant. This suggests that our solution have problems with maintaining its meaning in the synset extraction process. Although we have tried to find the correct meaning of each synset extracted from the concepts, the correct synset alone can not represent the meaning behind the word as it were in the original context. If we extract the phrase \emph{motor-driven butterfly valve} from \emph{Each bar is paired with a resonator tube that has a motor-driven butterfly valve at its upper end}, the facts that it is a part of a \emph{bar paired with a resonator tube} and it is placed \emph{at its upper end} is lost. \emph{Butterfly valve} in an empty context lacks the complete relation to the original concept.

%----------------------------------------------------------------------------------------

%\section{Theory}

%----------------------------------------------------------------------------------------

%\section{User input in the algorithm}

%----------------------------------------------------------------------------------------

\section{Application of algorithm}

\subsection{Human interpretation of blending results}

%----------------------------------------------------------------------------------------

\subsection{Different uses of the concept blending implementation}
One use case can be musicians that wants to explore new uses of their instrument. They can select their own instrument and a different instrument of choice for blending and the program will provide suggestions of properties to take from the other instrument and incorporate into their own instrument. Common examples of using elements from other instruments include the use of a violin bow on plucked instruments and cymbals or using a plectrum or percussive implements on bowed instruments.

%----------------------------------------------------------------------------------------

%\section{Obstacles}

%----------------------------------------------------------------------------------------

\section{Summary}

%----------------------------------------------------------------------------------------